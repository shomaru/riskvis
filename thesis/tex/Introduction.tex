%% ----------------------------------------------------------------
%% Introduction.tex
%% ---------------------------------------------------------------- 
\chapter{Introduction} \label{Chapter:Introduction}

Since the advent of the Internet, asynchronous threaded conversation platforms have been considered as the primary channel of information exchange. Surprisingly, the post and reply structure of threaded conversation has remained similar from the earlier email lists to the current Internet forums. Recently, an increasing number of social features such as participation statistics and contribution scores have been gradually integrated into Internet forums. The term social network refers to any form of relation between any types of social actors, potentially through mediating entities \citep{Brandes2008a}. For example, a typical social network in Internet forums is the reply network. When someone replies to a message, she creates a reply relation with the author of that message through this message. The Social Network Analysis (SNA) provides both a visual and a mathematical analysis of social networks.

The term Information Visualisation (or infovis) can be defined as a process that transforms abstract data into visual-spatial forms with the aid of computer software \citep{Card1999}. For instance, the visualisation most commonly used to analyse the reply network of Internet forums is the node-edge diagram, where the nodes represent contributors and the edges refer to the reply relation between two contributors. If someone replies the same contributor multiple times, a stronger edge should be created.

Defined in \citep{IS2009}, risk management is the identification, assessment, and prioritisation of risks. In the context of the reply network in Internet forums, there is a specific risk that a top contributor may leave the forum. Thus the whole process to deal with it can be described as follows:
\begin{enumerate}
	\item Risk identification: identifying the cause and consequence;\\
	\item Risk analysis: determining the risk attributes mainly impact and likelihood;\\
	\item Risk treatment: finding a solution that transfers or reduces the risk.
\end{enumerate}

As a part of the ROBUST project\footnote{http://robust-project.eu/}, this thesis claims that the risk management process can be enhanced with the aid of visualising the collaborative network in the context of Internet forums. The aim of this work is to help risk managers identify, analyse, and reduce the risk of experts leaving in a simple way. Which contributors are experts with in an Internet forum? What is the likelihood and impact of the risk of experts leaving? Who are successors to a leaving expert?

The outline of the thesis is as follows. Chapter~\ref{Chapter:RelatedWork} explores the previous work in the literature that developing software to visualise not only generic social networks but also specific Internet forums. Chapter~\ref{Chapter:Approach} introduces an iterative and incremental development methodology and highlights the development process which consists of requirements analysis, design, prototype, and evaluation of the forum visualisation tool. Also, several existing visualisation frameworks will be evaluated and one of them will be selected to develop the prototype in Chapter~\ref{Chapter:Prototype}. An analysis and modelling of the target Internet forum will be described in Chapter~\ref{Chapter:Requirements}, and then a set of requirements derived from the use cases will be also discussed. The design of the forum visualisation tool (Chapter~\ref{Chapter:Design}) defines the two different networks as well as their visual properties. Then the details in the user interface design of the tool will be given. The implementation of the prototype is specified in Chapter~\ref{Chapter:Prototype}. The architecture of the prototype as well as the modular approach will be described in more details. Chapter~\ref{Chapter:Evaluation} depicts the evaluation goals, participants, and methodology, and then presents the findings observed in the face-to-face interview.  This thesis is concluded with an outlook on future in Chapter~\ref{Chapter:Conclusions}.
